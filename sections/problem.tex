\section{The Problem}
Blockchain technology, despite its transformative potential, faces critical limitations that hinder its adoption in the emerging landscapes of artificial intelligence, edge computing, and space exploration. These challenges, rooted in the linear design of traditional ledgers, threaten to stall progress unless addressed by a radical new approach.

\begin{enumerate}
    \item \textbf{Scalability Constraints}: Conventional blockchains, such as Bitcoin~\cite{nakamoto2008bitcoin} and Ethereum~\cite{buterin2014ethereum}, rely on linear chains where each block appends sequentially, creating bottlenecks as transaction volumes grow. High demand leads to soaring fees and delays, with networks like Ethereum processing only tens of transactions per second under strain. This linear model cannot support the thousands of transactions per second required for real-time AI-driven markets or sprawling IoT networks, nor can it scale to the interplanetary distances where latency compounds the issue.

    \item \textbf{Limited Adaptability}: Today’s blockchains are ill-equipped for the specialized demands of autonomous AI agents, low-latency edge applications, and extraterrestrial finance. Most lack native mechanisms to handle subsecond transaction validation, critical for machine-to-machine economies or real-time gaming DApps. Furthermore, their rigid architectures struggle with the extreme communication delays inherent in space—3 to 24 minutes for Earth-to-Mars signals—rendering them impractical for decentralized systems beyond Earth.

    \item \textbf{Centralization Risks}: Efforts to scale blockchains often sacrifice decentralization, undermining their core ethos. Solutions like layer-2 rollups or static sharding~\cite{wang2019sharding} introduce trusted intermediaries or fixed partitions, concentrating control and creating single points of failure. Such compromises erode the permissionless, trustless foundation essential for universal adoption, particularly in autonomous AI ecosystems or uncharted space economies where centralized oversight is infeasible.
\end{enumerate}

These shortcomings—scalability bottlenecks, adaptability gaps, and centralization trade-offs—demand a reimagined blockchain architecture capable of supporting the decentralized future envisioned for Earth and beyond.